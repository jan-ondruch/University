\documentclass[11pt,a4paper,oneside]{article}
\special{papersize=210mm,297mm}
\usepackage[left=2cm,text={17cm, 24cm},top=3cm]{geometry}

\usepackage{times}
\usepackage[utf8x]{inputenc} 
\usepackage[czech]{babel}
\usepackage[IL2]{fontenc}

\usepackage{natbib}
\usepackage{dtklogos}

\usepackage{url}
\DeclareUrlCommand\url{\def\UrlLeft{<}\def\UrlRight{>} \urlstyle{tt}}
\bibliographystyle{czechiso}

\begin{document}

\begin{titlepage}
\begin{center}

{\Huge\textsc{{Vysoké učení technické v~Brně}}}\\
\bigskip
{\huge\textsc{{Fakulta informačních technologií}}}\\
\vspace{\stretch{0.382}}
{\LARGE{Typografie a publikování\,--\,4. projekt}}\\
\medskip
{\Huge{Bibliografické citace}}
\vspace{\stretch{0.618}}
\end{center}		
\Large{\today \hfill Jan Ondruch}
\end{titlepage}

	\section*{Typografie a její praktické užití}
	\LaTeX~je \texttt{velmi} kvalitní sázecí nástroj, se kterým můžeme vytvořit jednoduché dokumenty nebo také pomocí designových prvků technické a vědecké dokumenty. \LaTeX~je k~dispozici jako volný software.
	
	Z~tohoto plyne, že tento sázecí nástroj se může naučit bezplatně kdokoliv. Pro zvládnutí základních kroků, ale i~pokročilých sázecích technik je vhodná publikace \cite{J_Rybicka_LPZ}. Pokud spíše preferujete angličtinu, nabízí se poté \cite{S_Kottwitz_LBG}. V~dnešní době by však mohl někdo považovat investici do klasické knihy, popř. její výpujčky za zbytečnou. V~tomto případě osobně doporučuji stránku \cite{uvod_ds_latex}, která je velmi přehledná a~nabízí řadu názorných příkladů.
	
	Po zvládnutí základních technik a~práce se sázecím prostředím - např. \emph{Texmaker} se můžete pustit do pokročilejších technik \LaTeX u, které vám napomůžou strukturovat text způsobem, aby již na první pohled čtenáře zaujal. Zde se můžete inspirovat webovou stránkou \cite{typomil}. Naleznete zde např. pravidla kompozice textu nebo užitečné informace o~písmu a~jeho struktuře. Dalším pomocným zdrojem by se mohl pro vás stát práce \cite{zlaty_rez}.
	
	Jak moc přínosnou pro vás typografie může být dokazuje článek \cite{wong_typography}, který poukazuje na důležitost výběru správného fontu písma při zasílání žádosti o~práci. Jiným příkladem může být \cite{berleant_proposal}. \cite{dig_typografie} se zabývá tvorbou webových prezentací, v~nichž je písmo jedním ze stěžejních bodů celého konceptu práce a~němelo by zůstat v~žádném případě opomenuto. Například i~\cite{web_typografie} tvrdí, že je typografie i~přes moderní přístup k~navrhování a vytváření webových prezentací poněkud opomíjena. Fakt, že zobrazení textu (v~tomto případě dynamického) na moderních zařízeních je problematickou a~ne úplně vyřešenou oblastí nám abstrahuje práce \cite{dyn_text}.

\newpage
\renewcommand{\refname}{Literatura}
\bibliography{proj4}	
	
\end{document}