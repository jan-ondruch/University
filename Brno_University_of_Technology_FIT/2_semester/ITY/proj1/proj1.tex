\documentclass[11pt,a4paper,twocolumn,oneside]{article}
\usepackage[left=2cm,text={17cm, 24cm},top=2.5cm]{geometry}
\usepackage[utf8x]{inputenc} 
\usepackage[czech]{babel}
%\usepackage[IL2]{fontenc}
\usepackage{times}

\providecommand{\uv}[1]{\quotedblbase #1\textquotedblleft}
\newcommand{\subtitle}[1]{\Large{#1}}
\newcommand{\email}[1]{\large{#1}}

\begin{titlepage}

\title{Typografie a publikování\\1. projekt}
\author{Jan Ondruch\\xondru14@stud.fit.vutbr.cz}
\date{}

\end{titlepage}

\begin{document}

	\maketitle
	\section{Hladká sazba}

Hladká sazba je sazba z~jednoho stupně, druhu a řezu pí­sma sázená na stanovenou šířku odstavce. Skládá se z~odstavců, které obvykle začínají­ zarážkou, ale mohou být sázeny i bez zarážky -- rozhodují­cí­ je celková grafická úprava. Odstavce jsou ukončeny východovou řádkou. Věty nesmějí začínat číslicí.

Barevné zvýraznění­, podtrhávání­ slov či různé velikosti písma vybraných slov se zde také nepoužívá. Hladká sazba je určena především pro delší­ texty, jako je napří­klad beletrie. Porušení­ konzistence sazby působí v~textu rušivě a unavuje čtenářův zrak.

	\section{Smíšená sazba}

Smíšená sazba má o~něco volnější­ pravidla, jak hladká sazba. Nejčastěji se klasická hladká sazba doplňuje o~další řezy pí­sma pro zvýraznění­ důležitých pojmů. Existuje \uv{pravidlo}:

	\begin{quotation}
\textsc {Čí­m ví­ce druhů, řezů, velikostí, barev pí­sma a jiných efektů použijeme, tí­m profesionálněji bude  dokument vypadat. Čtenář tím bude vždy nadšen!}
	\end{quotation}

Tí­mto pravidlem se \underline{nikdy} nesmí­te ří­dit. Příliš časté zvýrazňování textových elementů  a změny {\Huge V}{\huge E}{\LARGE L}{\Large I}{\large K}{\normalsize O}{\small S}{\footnotesize T}{\scriptsize I} pí­sma {\large{jsou}} {\LARGE{známkou}} {\huge{\bfseries {amatérismu}}} autora a působí­ {\bfseries {\itshape velmi}} {\itshape rušivě}. Dobře navržený dokument nemá obsahovat ví­ce než 4 řezy či druhy pí­sma. {\ttfamily Dobře navržený dokument je decentní­, ne chaotický.}

Důležitým znakem správně vysázeného dokumentu je konzistentní použí­vání­ různých druhů zvýraznění­. To napří­klad může znamenat, že {\bfseries tučný řez} pí­sma bude vyhrazen pouze pro klíčová slova, {\slshape skloněný řez} pouze pro doposud neznámé pojmy a nebude se to míchat. Skloněný řez nepůsobí­ tak rušivě a používá se častěji. V~\LaTeX u jej sází­me raději pří­kazem \verb|\emph{text}| než \verb|\textit{text}|.

Smíšená sazba se nejčastěji používá pro sazbu vědeckých článků a technických zpráv. U~delší­ch dokumentů vědeckého či technického charakteru je zvykem upozornit čtenáře na význam různých typů zvýraznění­ v~úvodní­ kapitole.

	\section{České odlišnosti}

Česká sazba se oproti okolní­mu světu v~některých aspektech mí­rně liší­. Jednou z~odlišností je sazba uvozovek. Uvozovky se v~češtině použí­vají­ převážně pro zobrazení­ pří­mé řeči. V~menší­ míře se použí­vají­ také pro zvýraznění­ přezdí­vek a ironie. V~češtině se použí­vá tento typ \uv{uvozovek} namísto anglických "uvozovek".

Ve smíšené sazbě se řez uvozovek ří­dí­ řezem první­ho uvozovaného slova. Pokud je uvozována celá věta, sází­ se koncová tečka před uvozovku, pokud se uvozuje slovo nebo část věty, sází­ se tečka za uvozovku.

Druhou odlišností je pravidlo pro sázení­ konců řádků. V~české sazbě by řádek neměl končit osamocenou jednopí­smennou předložkou nebo spojkou (spojkou \uv{a} končit může při sazbě do 25 liter). Abychom \LaTeX u zabránili v~sázení­ osamocených předložek, vkládáme mezi předložku a slovo nezlomitelnou mezeru znakem \verb|~| (vlnka, tilda). Pro automatické doplnění vlnek slouží­ volně šiřitelný program {\itshape vlna} od pana Olšáka \footnote{Viz ftp://math.feld.cvut.cz/pub/olsak/vlna/.}.
\end{document}