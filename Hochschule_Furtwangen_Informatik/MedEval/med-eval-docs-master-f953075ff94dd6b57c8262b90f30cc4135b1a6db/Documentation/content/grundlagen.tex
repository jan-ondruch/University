\chapter{Grundlagen}
% Grundlagen limesurey
% Grundlagen Erweiterung des limesurvey
% Grundlagen Linux Server Verwalung
% Grundlagen Ansible
% Grundlagen Hardening
Soll das Projekt Med-Eval in zukünftigen Semesterprojekten oder Thesis Arbeiten erweitert werden, sollte zunächst die derzeit verwendeten Technologien betrachtet werden. Diese sind eine essenzielle Anforderung an die Studenten/innen.

\section{Kenntnisse und Fähigkeiten}
Eine zentrale Fähigkeit, welche für die Weiterführung benötigt wird, ist die Fähigkeit Server Systeme (ohne Grafische Oberfläche) installieren, bedienen sowie warten zu können. Diese Fähigkeit wird bereits im ersten Semester vermittelt. Zudem ist es sehr Ratsam sich mit der Orchestrierungssoftware Ansible vertraut zu machen. Diese Software ermöglicht es den Anwender schnell und mit wenig Aufwand vorkonfigurierte Abläufe neu deployen zu können. Die Platform Med-Eval wurde vollständig mittels Ansible automatisiert. 
Ratsam wäre zudem die Fähigkeit Risiken einer Software zu ermitteln und geeignete Gegenmaßnahmen ergreifen zu können. Hierbei kann jedoch in dem meisten Fällen auf Informationen im Internet zugriffen werden.
Weiterhin wäre es sehr ratsam sich mit Script Sprachen der Webentwicklung wie JavaScript oder Node.js vertraut zu machen, bzw. dort bereits Vorkenntnisse vorweisen zu können. Des weitern 2 Erweiterungen auf Systemebene mit Python geschrieben, weshalb es ratsam ist sich grundlegende Kenntnisse von Python anzueignen.

\section{Projektunterstützende Maßnahmen}
Soll das Projekt erfolgreich abgeschlossen werden, ist es Ratsam Projektunterstützende Maßnahmen zu ergreifen. Dies umfasst die Verwendung einer Versionsverwaltung zu nützen \url{https://github.com}. So können parallel Code Änderungen vorgenommen werden, eine Vollständige Historie des Codes erhalten so wie gezielt neue Funktionen in das existierende System aufgenommen werden.
Neben der Code Verwaltung ist ein weiterer wichtiger Punkt die Team Kommunikation. Diese sollte einfach und schnell umsetzbar sein. Um dies zu erreichen, kann auf die Teamkommunikationssoftware Slack \url{https://slack.com} genutzt werden. Diese Software wurde erfolgreich in Firmen und Organisationen wie der NASA sowie IBM erprobt.
\\
\section{Entwicklungsprozess}
Um einen möglichst reibungslosen Entwicklungsprozess zu erreichen sollte auf die Entwicklungskonzepte Test getriebene Entwicklung oder Agile zurückgegriffen werden.
Besonderes beim gewünschten Verwendungszweck der Plattform Med-Eval ist eine Entwicklung mit vielen Test sehr ratsam. So können bereits früh im Projekt wiederkehrende Fehler behoben oder gar vermieden werden. So viel in Regression Tests von Med-Eval auf, das die Download URL für jeden Releasewelchsel von LimeSurvey verändert wird. 
